\documentclass{article}

\usepackage{enumitem}
\usepackage{fontawesome}
\usepackage[margin=1in]{geometry}
\usepackage[utf8]{inputenc}
\usepackage{marvosym}
\usepackage{titlesec}
\usepackage{titling}
\usepackage{setspace}
\usepackage[dvipsnames]{xcolor}
\usepackage{hyperref}
\usepackage{amsmath}
\usepackage{amssymb}

%%%%%%%%%%%%%%%
% DEFINITIONS %
%%%%%%%%%%%%%%%

\onehalfspacing

%%%%%%%%%%%
% CONTENT %
%%%%%%%%%%%

\begin{document}

\title{Summary of the Mathematics of the Project}
\author{Arnav, Hussein, Youssef, Michael}

\maketitle

\section{Overview of the Project}

Using a Golang script to fetch the headers of files that were important, we created a database of files in the InterPlanetary File System offered by Estuary.
These summaries are then fed into a Cohere embed model that returns a 4096 feature vector embedding of each file.
In this embedding scheme, vectors that have a lower magnitude of difference are closer to each other and share greater similarities.
To the end of identifying and visualizing the similar documents, feature reduction was implemented to condense the features down to 3 features with as little information loss as possible.
Following this, the generated data points 

\section{Feature Reduction}

The method used to do feature reduction is Principal Component Analysis.
The large dataset of multiple variables first undergoes a change of coordinate basis to a basis where three of the vectors are the principal components of the dataset, which is to say that there is the greatest variation of data encapsulated in those dimensions.
The new coordinates for an arbitrary vector after the feature reduction is simply the component of the vector along each of the principal components.

This is equivalent to interpreting the new vector as a projection of the multidimensional vector space onto the space that contains the 3 principal components. 

\section{Clustering}

Clustering the new points in $\mathbb{R}^3$ was achieved by using a kNN algorithm where k was 2.
Using the kNN to cluster points that were nearby in the 3D space meant that the only files that are connected are those that are similar in content.
Then, files are expressed as nodes of a graph with edges between other nodes (files) if they are related.
A force simulation is then applied, allowing the user to visually explore the data structure while maintaining the proximity and connections of clusters of nodes.

\end{document}
